\documentclass[10pt,a4paper,french]{article} % Specify the font size (10pt, 11pt and 12pt) and paper size (letterpaper, a4paper, etc)

\usepackage{graphicx} % Required for including pictures
%\usepackage{microtype} % Improves typography

\usepackage{babel}
\usepackage[T1]{fontenc} % Required for accented characters
\usepackage{hyperref}
% Create a new command for the horizontal rule in the document which allows thickness specification
\PassOptionsToPackage{svgnames}{xcolor}
\usepackage[most]{tcolorbox}


% police sans serif 
\usepackage{helvet}
\renewcommand{\familydefault}{\sfdefault}
%marges 

\usepackage{geometry}
 \geometry{
 a4paper,
 total={170mm,257mm},
 left=20mm,
 top=20mm,
 }



\begin{document}



Modèle de sol 

Objectifs du modèle :


Modéliser les variations de rendement agricole induites par : 

\begin{itemize}
	\item l'amendement du sol par fumure animale  (fertilité instantanée)
	\item la paille laissée au sol (fertilité instantanée)
	\item la qualité du sol (fertilité long terme)
	\item l'érosion de la fertilité (fertilité long terme)
\end{itemize} 

\vspace{0.5cm}
Le modèle simplifié fourni par Arthur modélise la variation d'azote dans le sol. 
Les compléments fournis par Antoine introduisent l'effet de la qualité du sol sur l'azote disponible (minéralisation) et l'érosion de l'azote (lixiviation). 



\section{L'azote disponible}



La première équation donne l'azote (en kg) disponible à l'année $n$ sur la parcelle $p$ : $N_{available}^{n}(p)$

\begin{equation}
N_{available}^{n}(p)=N_{soil}^{n}(p)+N_{air}^{n}(p)+N_{manure}^{n}(p)+N_{faid}^{n}(p)-N_{erosion}(p)
\end{equation}


\subsection{Azote issu de l'air}




\begin{equation}
N_{air}^{n}(p) = A(p) * 0.002
\end{equation}

\subsection{Azote issu du sol}


L'azote issu du sol est une somme de deux termes . Le premier est la \textbf{mineralisation}, fonction de l'aire $A(p)$ de $p$ le second est l'effet de la \textbf{qualité du sol} à l'année $n$:  $QS^{n}(p)$

\begin{equation}
N_{soil}^{n}(p) = 0.0012 * A(p) + QS^{n}(p)
\end{equation}



\subsubsection{Qualité du sol}

La qualité du sol à l'année $n$ est fonction de la biomasse produite sur la parcelle $p$ , en kg de végétal ($Crop$) cultivé l'année précédente ($n-1$)


\begin{equation}
QS^{n}(p) = Biomass(Crop^{n-1},p) * NQuantityRatio(Crop) * ResidueRatio(Crop) \\
\end{equation}

Avec  :

\begin{itemize}
	\item $Biomass(Crop^{n-1},p)$  la biomasse produite l'année dernière sur $p$ , en kg de végétal
	\item $NQuantityRatio(Crop)$ le pourcentage d'azote de la plante 
	\item $ResidueRatio(Crop)$  la quantité de résidus (paille) de la plante laissée sur  $p$ après la récolte
\end{itemize}



\begin{tcolorbox}[noparskip,
    colback=LightGreen,colframe=DarkGreen,%
    colbacklower=LimeGreen!75!LightGreen,%
    title=Question]
La biomasse , c'est le poids de toute la plante ? (i.e. pas que la partie  récoltée)
Faut-il compter les racines ? 
\end{tcolorbox}






La quantité de résidus laissés sur la parcelle varie selon les pratiques agricoles.
Au maximum, le résidu laissé correspond à la proportion de plante non récoltée (la paille).



\begin{table}[h!]
\begin{tabular}{|c|c|c|}
\hline
\textbf{Crop}          & \textbf{NQtyRatio}    & \textbf{Max residue ratio} \\ \hline
Mil                    & 1.5\%                 & 0.7                        \\ \hline
Peanut                 & 3\%                   & 0.666                      \\ \hline
Fallow                 & 2\%                   & 1                          \\ \hline
\end{tabular}
\end{table}



\begin{tcolorbox}[noparskip,
    colback=LightGreen,colframe=DarkGreen,%
    colbacklower=LimeGreen!75!LightGreen,%
    title=Question]

La jachère n'est pas récoltée (parfois très marginalement pour des reserves de fourrage, qu'on néglige ) \textbf{mais} elle est paturée par le troupeau $\rightarrow$ Faut-il  tenir compte de l'herbe pâturée (=mangée par le troupeau et donc emportée hors de $p$) et modifier le ratio de résidus de Jachère dans l'effet de sa biomasse sur l'azote du sol ?  Si on met 1, c'est que toute la biomasse reste là...
\end{tcolorbox}




\subsubsection{Production de biomasse}


Nous avons des rendements nominaux pour le Mil (600kg/ha, min. 450kg/ha , max. 1T/ha), pour l'arachide ( 450 kgs/ha, min. 300kg/ha max 600kg/ha) et pour la jachère (450kg/ha de fresh weeds, [Scriban ODD model] ).

Ces rendements sont à dire d'acteurs et ne nous sont pas données pour chaque combinaison de pratiques (engrais,fumure).

Comme on connaît les proportions de  produit/coproduits , on peut calculer la biomasse de végétal par hectare étant donné le rendement en produit : 

Mil : 30\% de produit ,  70\% de paille \\
Arachide : 1.5 fois plus de coproduit (fanes) que de gousses

\begin{equation}
Biomasse(Mil) =  \frac{Rendement(Mil)}{0.3}
\end{equation}

\begin{equation}
Biomasse(Arachide) =  2.5 \times Rendement(Arachide)
\end{equation}\\


\subsubsection{Application numérique:} 

Pour un rendement de 600kg/ha de Mil graines, production de biomasse de 2T/ha (0.2kg/m²)\\
Pour un rendement de 450kg/ha d'arachide, production de biomasse de 1.125T/ha (0.11kg/m²)
Pour un rendement de 475kg/ha d'herbe en jachère (0.0475kg/m²) 



On laisse toute la paille et les racines sur une parcelle $p$ d'aire $A(p)=100m²$


$QS^{n}(Mil, p)= 0.2 A(p)* 0.015 * 0.7 = 0,0021* A(p)= 210g$ d'azote issu de la biomasse de Mil dans $p$ \\
$QS^{n}(Peanut, p)= 0.11 A(p) * 0.03 * 0.6 = 0.21978 = 220g$ d'azote issu de la biomasse d'Arachide  dans $p$\\
$QS^{n}(Fallow, p)= 0.0475 A(p)* 0.02 * 1 = 0.095 = 10g$ d'azote issu de la biomasse de jachère  dans $p$ \\



\subsection{Azote issu du fumier}






\subsection{Azote issu des Faidherbia}


4kg par arbre [Scriban ODD model]


\subsection{Azote de l'année précédente}



Après la récolte , et en fonction des pratiques agricoles, il reste une fraction d'azote 




\begin{equation}
  NRF(N_{available}(Mil))=\left\{
                \begin{array}{ll}
                  0.25 \ if \ N_{available} < 18\\
                0.501 \ if \ 18 \leq N_{available} \leq 83  \\
                  1 \ otherwise
                \end{array}
              \right\} 
 \end{equation} 

 \begin{equation}
   NRF(N_{available}(Fallow))=\left\{
                \begin{array}{ll}
                  0.25 \ if \ N_{available} < 10\\
                -1.2179 \ if \ 10 \leq N_{available} \leq 50  \\
                  1 \ otherwise
                \end{array}
              \right\}
  \end{equation} 
  
  

\begin{equation}
QS^{n}(Crop) = Yield(Crop) * NQuantityRatio(Crop) * ResidueRatio(Crop) \\
QS^{n}(Mil)= 0,06 * 0.015 * 0.7 = 0,00063\\
QS^{n}(Peanut)= 0,04 * 0.03 * 0.6 = 0,00084\\
QS^{n}(Fallow)= 0.0475* 0,02 * 1 = 0,00095\\
\end{equation}
  
  \begin{equation}
  N_{mineral}(p) = OM_{sand} * A(p)
  \end{equation}
  
\begin{equation}
Yield(Crop, N_{available}^{n}) = N_{available}^{n}(p) * \alpha(Crop) 
\end{equation}


Ce qu'il nous manque : 

La conversion de la masse d'azote à la biomasse : 


\begin{equation}
N_{available}^{n}(p)=N_{remaining}^{n-1}(p) + (1 - NRF(N_{available}^{n}(p))) * N_{available}
\end{equation}


\end{document}