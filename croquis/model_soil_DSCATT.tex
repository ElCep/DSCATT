\documentclass[10pt,a4paper,french]{article} % Specify the font size (10pt, 11pt and 12pt) and paper size (letterpaper, a4paper, etc)

\usepackage{graphicx} % Required for including pictures
%\usepackage{microtype} % Improves typography

\usepackage{babel}
\usepackage[T1]{fontenc} % Required for accented characters
\usepackage{hyperref}
% Create a new command for the horizontal rule in the document which allows thickness specification
\PassOptionsToPackage{svgnames}{xcolor}
\usepackage[most]{tcolorbox}


% police sans serif 
\usepackage{helvet}
\renewcommand{\familydefault}{\sfdefault}
%marges 

\usepackage{geometry}
 \geometry{
 a4paper,
 total={170mm,257mm},
 left=20mm,
 top=20mm,
 }



\begin{document}



Modèle de sol 

Objectifs du modèle :


Modéliser les variations de rendement agricole induites par : 

\begin{itemize}
	\item l'amendement du sol par fumure animale  (fertilité instantanée)
	\item la paille laissée au sol (fertilité instantanée)
	\item la qualité du sol (fertilité long terme)
	\item l'érosion de la fertilité (fertilité long terme)
\end{itemize} 

\vspace{0.5cm}
Le modèle simplifié fourni par Arthur modélise la variation d'azote dans le sol. 
Les compléments fournis par Antoine introduisent l'effet de la qualité du sol sur l'azote disponible (minéralisation) et l'érosion de l'azote (lixiviation). 



\section{L'azote disponible}



La première équation donne l'azote (en kg) disponible à l'année $n$ sur la parcelle $p$ : $N_{available}^{n}(p)$

\begin{equation}
N_{available}^{n}(p)=N_{soil}^{n}(p)+N_{air}^{n}(p)+N_{manure}^{n}(p)+N_{faid}^{n}(p)-N_{erosion}^n(p) + N_{remaining}^{n-1}
\end{equation}


\subsection{Azote issu de l'air}


Azote de l'air fixé par le sol

\begin{equation}
N_{air}^{n}(p) = A(p) * 0.002
\end{equation}

\subsection{Azote issu du sol}


L'azote issu du sol est une somme de deux termes . Le premier est fonction de l'aire $A(p)$ de $p$ le second est l'effet de la \textbf{qualité du sol} (fixation par les micro-organismes? ) à l'année $n$:  $QS^{n}(p)$

\begin{equation}
N_{soil}^{n}(p) = 0.0012 * A(p) + QS^{n}(p)
\end{equation}



\subsubsection{Qualité du sol}

La qualité du sol à l'année $n$ est fonction de la biomasse produite sur la parcelle $p$ , en kg de végétal ($Crop$) cultivé l'année précédente ($n-1$)


\begin{equation}
QS^{n}(p) = Biomass(Crop^{n-1},p) * NQuantityRatio(Crop) * ResidueRatio(Crop) \\
\end{equation}

Avec  :

\begin{itemize}
	\item $Biomass(Crop^{n-1},p)$  la biomasse produite l'année dernière sur $p$ , en kg de végétal
	\item $NQuantityRatio(Crop)$ le pourcentage d'azote de la plante 
	\item $ResidueRatio(Crop)$  la quantité de résidus (paille) de la plante laissée sur  $p$ après la récolte
\end{itemize}



\begin{tcolorbox}[noparskip,
    colback=LightGreen,colframe=DarkGreen,%
    colbacklower=LimeGreen!75!LightGreen,%
    title=Question]
La biomasse , c'est le poids de toute la plante ? (i.e. pas que la partie  récoltée)
Faut-il compter les racines ? 
\end{tcolorbox}






La quantité de résidus laissés sur la parcelle varie selon les pratiques agricoles.
Au maximum, le résidu laissé correspond à la proportion de plante non récoltée (la paille).



\begin{table}[h!]
\begin{tabular}{|c|c|c|}
\hline
\textbf{Crop}          & \textbf{NQtyRatio}    & \textbf{Max residue ratio} \\ \hline
Mil                    & 1.5\%                 & 0.7                        \\ \hline
Peanut                 & 3\%                   & 0.666                      \\ \hline
Fallow                 & 2\%                   & 1                          \\ \hline
\end{tabular}
\end{table}



\begin{tcolorbox}[noparskip,
    colback=LightGreen,colframe=DarkGreen,%
    colbacklower=LimeGreen!75!LightGreen,%
    title=Question]

La jachère n'est pas récoltée (parfois très marginalement pour des reserves de fourrage, qu'on néglige ) \textbf{mais} elle est paturée par le troupeau $\rightarrow$ Faut-il  tenir compte de l'herbe pâturée (=mangée par le troupeau et donc emportée hors de $p$) et modifier le ratio de résidus de Jachère dans l'effet de sa biomasse sur l'azote du sol ?  Si on met 1, c'est que toute la biomasse reste là...
\end{tcolorbox}




\subsubsection{Production de biomasse}


Nous avons des rendements nominaux pour le Mil (600kg/ha, min. 450kg/ha , max. 1T/ha), pour l'arachide ( 450 kgs/ha, min. 300kg/ha max 600kg/ha) et pour la jachère (450kg/ha de fresh weeds, [Scriban ODD model] ).

Ces rendements sont à dire d'acteurs et ne nous sont pas données pour chaque combinaison de pratiques (engrais,fumure).

Comme on connaît les proportions de  produit/coproduits , on peut calculer la biomasse de végétal par hectare étant donné le rendement en produit : 

Mil : 30\% de produit ,  70\% de paille \\
Arachide : 1.5 fois plus de coproduit (fanes) que de gousses

\begin{equation}
Biomasse(Mil,p) =  \frac{Rendement(Mil,p)}{0.3}
\end{equation}

\begin{equation}
Biomasse(Arachide,p) =  2.5 \times Rendement(Arachide,p)
\end{equation}\\


\subsubsection{Application numérique:} 

Pour un rendement de 600kg/ha de Mil graines, production de biomasse de 2T/ha (0.2kg/m²)\\
Pour un rendement de 450kg/ha d'arachide, production de biomasse de 1.125T/ha (0.11kg/m²)
Pour un rendement de 475kg/ha d'herbe en jachère (0.0475kg/m²) 



On laisse toute la paille et les racines sur une parcelle $p$ d'aire $A(p)=100m²$


$QS^{n}(Mil, p)= 0.2 A(p)* 0.015 * 0.7 = 0,0021* A(p)= 210g$ d'azote issu de la biomasse de Mil dans $p$ \\
$QS^{n}(Peanut, p)= 0.11 A(p) * 0.03 * 0.6 = 0.21978 = 220g$ d'azote issu de la biomasse d'Arachide  dans $p$\\
$QS^{n}(Fallow, p)= 0.0475 A(p)* 0.02 * 1 = 0.095 = 10g$ d'azote issu de la biomasse de jachère  dans $p$ \\




\subsubsection{Initialisation du modèle}


L'année $n=0$ , il n'y a pas de récole précédente pour  avoir une quantité d'azote issue des effets de précédent : $QS^n(p)=0$


\begin{tcolorbox}[noparskip,
    colback=LightGreen,colframe=DarkGreen,%
    colbacklower=LimeGreen!75!LightGreen,%
    title=Question]


\end{tcolorbox}






\subsection{Azote issu du fumier}



$N_{manure}^n(p)= ManureMass(p) * 0.0238$


[Scriban ODD model]




\subsection{Azote issu des Faidherbia}


4kg par arbre [Scriban ODD model]


\subsection{Azote de l'année précédente}



$N_{remaining}^n$


Après la récolte , et en fonction des pratiques agricoles, il reste sur la parcelle l'azote qui n'a pas été emporté par la récolte. (l'azote minéralisé ?)
Cette fraction d'azote est déjà calculée à l'étape de la qualité du sol, mais pour la récolte de l'année passée . (fonction de la biomasse produte , du pourcentage d'azote dans la plante, et du pourcentage de la plante qu'on laisse sur place) 



\begin{tcolorbox}[noparskip,
    colback=LightGreen,colframe=DarkGreen,%
    colbacklower=LimeGreen!75!LightGreen,%
    title=Question]
On n'avait pas identifié que l'azote restant apparaissait \textbf{deux fois} dans le modèle : une fois dans $N_{remaining}$ (la quantité d'azote qui reste de l'année passée), et une autre fois dans le terme $QS$, qui utilise la biomasse produite l'année passée et la proportion de végétal laissée sur place. Est-ce qu'il faut le prendre deux fois en compte ? (je ne pense pas)
\end{tcolorbox}


\section{Conversion Azote $\rightarrow$ Plante}



Les équations de "NRF" donnent la part d'azote disponible convertie en végétal selon les équation suivantes , issue de [Scriban ODD model]  , repris de la thèse de Myriam Grillot 


Il s 'applique normalement à une au rendement limité par l'eau,  aux équations définies par morceaux de façons similaires à celles des NRF  ci-dessous. 

\begin{equation}
  NRF(N_{available}^n,Mil)=\left\{
                \begin{array}{ll}
                  0.25 \ if \ N_{available} < 18\\
                0.501 \ if \ 18 \leq N_{available} \leq 83  \\
                  1 \ otherwise
                \end{array}
              \right\} 
 \end{equation} 

\begin{equation}
   NRF(N_{available}^n,Peanut)=1
  \end{equation} 
 

 \begin{equation}
   NRF(N_{available}^n,(Fallow))=\left\{
                \begin{array}{ll}
                  0.25 \ if \ N_{available} < 10\\
                -1.2179 \ if \ 10 \leq N_{available} \leq 50  \\
                  1 \ otherwise
                \end{array}
              \right\}
  \end{equation} 
  
  




Pour calculer la récolte (en kgs de végétal) , il faut convertir une partie de l'azote disponible en végétal.
Ce qu'il nous manque c'est le coefficient $\alpha_{N \rightarrow plant}$  \emph{toutes choses égales par ailleurs}. Il n'est pas sur qu'un tel coefficient existe.




\begin{tcolorbox}[noparskip,
    colback=LightGreen,colframe=DarkGreen,%
    colbacklower=LimeGreen!75!LightGreen,%
    title=Question/Problème]

Du point de vue calculatoire, on peut le dériver à partir des rendement nominaux fournis , comme pour la biomasse. Est-ce valable du point de vue agro ? 

\end{tcolorbox}


Exemple numérique : 
sur une parcelle $p$ de 100m², sans fumier, avec 1 faidherbia, l'année 0, on néglige l'érosion.

Rendement nominal du Mil : 600kg/ha (graines) , soit 6kgs de graines sur $p$

\begin{eqnarray*}
N_{available}^{0}(p)&=&N_{soil}^{n}(p)  +N_{air}^{n}(p)+N_{manure}^{n}(p)+N_{faid}^{n}(p)-N_{erosion}^n(p) + N_{remaining}^{n-1} \\
                    &=& 0.0012 * A(p) + QS^{n}(p) + A(p) * 0.002 +0                +4              -0                + 0  \\
                    &=&  0.0012 * 100 + 0 + 100 * 0.002 +0+4-0 + 0  \\
                    &=&  0.12 + 0.2 +4  \\
                    &=& 4.32
\end{eqnarray*}

On cultive  du Mil, et on dispose de 4.32kg d'azote sur $p$

On est en dessous de 18kg de N par ha, donc on applique un NRF de 0.25 : $\frac{1}{4}$ de ce N est converti en plante : environ 1kg de mil (matière sèche de toute la plante)

Sur ce kilo de plante Mil , 30\% est de la graine soit 300g de Mil produit

On est loin du rendement nominal du Mil annoncé par les agros pasteurs : 600kg de graines par ha, soit 2T de plante Mil 







\subsection{Azote restant après récolte}


\begin{equation}
N_{available}^{n}(p)=N_{remaining}^{n-1}(p) + (1 - NRF(N_{available}^{n}(p))) * N_{available}
\end{equation}


\end{document}