\chapter{Conclusion}

Le territoire fait face à un phénomène d’éclatement des cuisines. Le concept de cuisine regroupe les personnes qui mangent ensemble et donc qui participent à l’alimentation. Du fait de l’imbrication des systèmes de solidarité (et du lien avec la lignée maternelle?)([paul] casse gueule la lignée ici) , la propension des acteurs à participer à l’alimentation de la cuisine est réduite aux obligations. Pour nos participants, les femmes participent très fortement à l’éclatement des cuisines.

«Avec l’éclatement des cuisines, il y a de moins en moins de place pour les fainéants. Tout le monde doit travailler à fond. »


Le prêt de terre est historiquement accompagné d’un cadeau en nature, mais aujourd’hui, cette pratique évolue vers de la location.

«Il y a de la solidarité en face de l’insuffisance»
En cas de maladie d’un paysan, le village se mobilise pour cultiver et récolter.
De la nourriture est glissée sous la porte, les greniers sont remplis de nuit , sans le dire.
la discrétion est importante
