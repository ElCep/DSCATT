\chapter{Conclusion}

Le territoire fait face à un phénomène d’éclatement des cuisines. Le concept de cuisine regroupe les personnes qui mangent ensemble et donc qui participent à l’alimentation. Du fait de l’imbrication des systèmes de solidarité (et du lien avec la lignée maternelle?)([paul] casse gueule la lignée ici) , la propension des acteurs à participer à l’alimentation de la cuisine est réduite aux obligations. Pour nos participants, les femmes participent très fortement à l’éclatement des cuisines.

«Avec l’éclatement des cuisines, il y a de moins en moins de place pour les fainéants. Tout le monde doit travailler à fond. »


Le prêt de terre est historiquement accompagné d’un cadeau en nature, mais aujourd’hui, cette pratique évolue vers de la location.

«Il y a de la solidarité en face de l’insuffisance»
En cas de maladie d’un paysan, le village se mobilise pour cultiver et récolter.
De la nourriture est glissée sous la porte, les greniers sont remplis de nuit , sans le dire.
la discrétion est importante



\section{Modélisation et simulation de la survivance de la jachère}


Nous avons listés dans ce rapport l'ensemble des structures relationnelles dans lesquelles s'inscrivent les interactions que décrivent les acteurs.    
Sur la base des informations recueillies, nous avons entrepris de modéliser le système d'interactions sociales du village de Diohine de façon incrémentale, c'est-à-dire en controlant la complexification du modèle, de façon à n'inclure dans celui-ci les seuls mécanismes nécessaires à étudier une certaine question. Par exemple, un module spécifique devra être développé si l'on s'intéresse à la gestion des conflits, mais ce module ne sera pas mobilisé  si on s'intéresse à l'effet de la variation de la quantité d'engrais chimiques reçus à Diohine sur le rendement des cultures d'arachide.

\subsection{Socle commun des modèles}

Il s'agit de modéliser le système démographique et agricole du village, par les mécanismes suivants  : 
\begin{itemize}
\item découpage et attribution de l'espace agricole du village aux cuisines 
\item culture des parcelles
\item cycle des cultures et de la jachère
\item pâturage du bétail et fumure des terres
\item dynamique démographique du village
\end{itemize} 


Ce socle commun est initialisé à l'aide des faits recueillis auprès des participants de l'atelier  et de données  exogènes provenant d'autres travaux de recherche, en particulier celles qui nous permettent de simuler la subsistance du village par leur activité agricole :


\begin{itemize}
\item relevés démographiques du village de Diohine sur les 20 dernières années
\item rendement des parcelles 
\item quantités de fumure par tête de bétail et effet sur la fertilité des sols
\end{itemize}


Les observables de ce modèle initial concernent l'influence mutuelle qui s'exerce entre la démographie du village et la quantité de nourriture issue des récoltes. 

\subsection{Modules envisagés }

Le premier module envisagé est celui qui modélise l'éclatement et l'absorption des cuisines



\subsection{Résilience et viabilité du système}
