%Ne pas numéroter cette partie
\part*{Annexes}
%Rajouter la ligne "Annexes" dans le sommaire
\addcontentsline{toc}{part}{Annexes}

\section{Conclusion}

Le territoire fait face à un phénomène d’éclatement des cuisines. Le concept de cuisine regroupe les personnes qui mangent ensemble et donc qui participent à l’alimentation. Du fait de l’imbrication des systèmes de solidarité (et du lien avec la lignée maternelle?)([paul] casse gueule la lignée ici) , la propension des acteurs à participer à l’alimentation de la cuisine est réduite aux obligations. Pour nos participants, les femmes participent très fortement à l’éclatement des cuisines.

«Avec l’éclatement des cuisines, il y a de moins en moins de place pour les fainéants. Tout le monde doit travailler à fond. »


Le prêt de terre est historiquement accompagné d’un cadeau en nature, mais aujourd’hui, cette pratique évolue vers de la location.

«Il y a de la solidarité en face de l’insuffisance»
En cas de maladie d’un paysan, le village se mobilise pour cultiver et récolter.
De la nourriture est glissée sous la porte, les greniers sont remplis de nuit , sans le dire.
la discrétion est importante


\section{Biblio pour le moment , à refaire en latex}

Becu, N., Bommel, P., Botta, A., Le Page, C., Perez, P., 2010. Les téchnologies mobilisées pour l’accompagnement, in: Etienne, M. (Ed.), La modélisation d’accompagnement une démarche participative en appui au développement durable. Quae éditions, Versailles, pp. 183–201.

Etienne, M., Du Toit, D., Pollard, S., 2011. ARDI: A Co-construction Method for Participatory Modeling in Natural Resources Management. Ecology and Society 16. [https://doi.org/10.5751/ES-03748-160144](https://doi.org/10.5751/ES-03748-160144)

----

\section{ Fin du document }

----

\subsection{ TODO sur le DOT}



- ouvrir le .dot dans R avec sna -> enregistrer en `.gv` on l'ouvre tout seul dans Rstudio. 
%[La doc de rendu est pas mal](https://rich-iannone.github.io/DiagrammeR/graphviz_and_mermaid.html)
- enrichir le graphe avec des attributs dans R ou à la main dans le fichier en utilisant la synthaxe de DiagrammR. par exemple `@@1` pour définir un attribut 
- visualiser avec diagrammeR
- Pour le rendu : utilisation de la substitution de diagrammeR pour changer les couleurs les formes etc.  des noeuds et des arêtes

\subsection{ Le workflow}
- Defintion de class de noeud
- Extrait les sous graph avec igraph
- enrichissemement des attribut du graph genre color etc.
renvoyer en `.gv`

Pour chaque sous graphe du graphe PARDI 

-  griser les noeuds non-impliqués dans un sous-graphe de façon à conserver la topologie/spatialisation du gros graphe et ne faire apparaître que les noeuds impliqués dans le sous-système



\section{TODO rédaction du texte }

\subsection{TODO GRAPHE}

\subsection{ TODO }

Typologie de trois catégories : une par sous graphe 

Rassembler les éléments du terrain (notes, souvenirs , soit la biblio , soit l'expérience perso) dans chacun des trois sous graphe.

Rédiger de façon descriptive

Enrichir/Augmenter la description avec de l'analytique  : caractérisation des processus (e.g. Ici les acteurs mobilisent leur capital social), des postures des acteurs, des résultats de l'action collective 


Illustrations diverses

Meta-discours : contextualisation , intro , conclusion et des encarts sur des termes spécifiques pour se faire bien comprendre par tous (e.g. Saltigué et stygmergie sont deux termes qu'il sera bon d'expliciter à tous )






\subsection{Journal de bord du modèle de simulation}


\subsubsection{Note pour un jour}
 - Les lois en droit positif on l'oublie trop souvent servent à maintenir la paix sociale.
 - Spinoza pose en effet qu'il suffit de ne pas comprendre pour moraliser. Et c'est ce à quoi servent les lois. À gérer les problèmes des gens qui ne veulent pas comprendre. "Si Adam ne comprend pas la règle du rapport de son corps avec le fruit, il entend la parole de Dieu comme une défense. Bien plus, la forme confuse de la loi morale a tellement compromis la loi de la nature que la philosophie ne doit pas parler de loi de la nature, mais seulement de vérité éternelles "  p. 35, "les lois morale ou sociale ne nous apportent aucune connaissance, elle ne fait rien connaitre". Au pire elle empêche la formation de la connaissance (loi des tyrans). Au mieux elle prépare la connaissance et la rend possible (loi d'Abraham). Entre ces deux extrêmes, elle supplée à la connaissance chez ceux qui n'en sont pas capables en raison de leurs modes d'existence (loi de moise)."
>[name= Etienne] J'ai l'impression que tout le système traditionnel à Diohine est tourné vers une résolution de conflit où la seule chose qui est donnée aux acteurs c'est des indications de posture (tu ne dois pas faire de tort...-> injonction morale/ethique -> ligne de conduite serere), Donc on les pousse à comprendre la source du conflict (mauvais de Spinoza)parce qu'il y a une injonction a l'action (une action de réparation). 
>
 Changement de plan/d'arène juridique : Quand ils ne sont pas parvenus à une résolution locale du conflit, on entre dans le droit positif (dur et sans empathie -> paul Sene) et là il y a des lois qui font qu'on a plus besoin de comprendre le source du mauvais.--> un lien avec un verbatim en [J4](\url{https://hackmd.openmole.org/qhPAjsJGRbiOQYIItbwPww#J4---21-octobre}) "Tout le processus est là pour éviter la dureté des lois. “*au niveau du sous-préfet, il n’y a plus de sentiments, c’est la légalité dans toute sa froideur*”"
>[name= Paul] Ne pas vouloir qu'on casse la paix sociale ça veux pas dire qu'on cherche a la maintenir. Les mécanismes d'exclusion, sont une réaction au comportement délétère et pas une proaction en faveur de son épanouissement. La loi intervient "en négatif" pour se débarasser de ce qui met à mal la paix sociale. Une question pour Philippe Karpe : le capital de paix sociale est il croissant par nature ? 
>[name= Etienne] Quand une personne va en brousse, c'est pour se faire oublier donc la paix social


La théorie de l'économie des conventions [boltanski et thévenot]  semble bien adaptée pour lire la paix sociale de Diohine: (source wikipedia) Celui-ci part de l'idée que pour qu'il y ait échange, coordination, coopération entre des agents, il faut qu'il y ait des *conventions* entre les personnes concernées ; c’est-à-dire un *système d'attentes réciproques* entre les personnes sur leurs comportements.

> [name=Paul ] ces attentes peuvent être dans au moins trois cités : la cité domestique , cité civique et peut être cité par projets dans une moindre mesure pour des décision ponctuelles : orientation spcéifique après la première chasse, creuser un puits, établir une banque etc. avec les acteurs qui ne sont pas agro pasteurs 
> Ca sera surtout utile pour la gestion des conflits avec les différents niveaux de la colonnes de spération du conflit  cf la page wikipedia : :     Il survient une controverse dans une même cité. Pour la clore, on recourt à un principe supérieur commun. Car les personnes engagées dans une même cité ont un même système d'équivalence, ils se déplacent dans une grandeur identique. Les objets sont identifiés et hiérarchisés de manière compatible.
    Il peut coexister des cités différents sans discordes. Mais dans ce cas l'équilibre reste provisoire.
    Il peut survenir un différend entre des cités. La discorde doit, pour être clarifiée, être rapportée à une cité et une seule. Elle peut également être résolue par un arrangement, les partenaires se mettent localement d'accord sur une transaction. Enfin, les acteurs peuvent arriver à un compromis, et dans ce cas, ils réunissent plusieurs cités à travers un bien commun.
   \url{ https://www.cnam.fr/servlet/com.univ.collaboratif.utils.LectureFichiergw?ID_FICHIER=1295877017868}


>[name=Paul] ça faisait longtemps qu'on avait pas fait une métaphore faiblarde : \url{https://fr.wikipedia.org/wiki/Colonne_(s%C3%A9paration)} 
Colonne de rectification du conflit : plusieurs étages, plusieurs niveaux de résolution du conflit -> les plateaux de la colonne

