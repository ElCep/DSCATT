\renewcommand{\abstractnamefont}{\normalfont\Large\bfseries}
%\renewcommand{\abstracttextfont}{\normalfont\Huge}

\begin{abstract}
\hskip7mm

\begin{spacing}{1.3}

  Dans le cadre du projet DSCATT, nous avons animé plusieurs ateliers à Diohine sur le territoire de l'observatoire IRD de Niakhar. Ces ateliers s'inscrivaient dans le WP4 du projet "\textit{Co-design and evaluation of best-farm practices for soil C sequestration}". Une première série d'ateliers ont été mener en début d'année 2021 avec pour objectif d'identifier les aspirations des acteurs de la zone pour les accompagner vers de la planification territoriale.\cite{perrotton_definition_2021} Parmis ces aspirations, identifiés par les acteurs locaux au cours du processus ACADRI, nous avons choisie de mener un travail plus attentif a l'une d'elles : "une sécurisation de la gestion foncière traditionnelle".

  Dans ce travail, nous avons fait le choix et le pari de questionner les enjeux de séquestration du carbone à travers la gestion foncière par un exercice de modélisation conceptuelle. Nous considérons en effet que "parce qu’elle simplifie, la modélisation est un puissant instrument de mise en évidence du général. Mais elle permet aussi d’orienter la recherche des spécificités, donc de retrouver le particulier, et d’en donner une image plus efficace que celle qui résulte de l’accumulation de monographies sans idée directrice". \cite{durand-dastes_particulier_1991}

  En modélisant avec les acteurs, les dynamiques et les interactions liées à la gestion foncière, nous voulons plus spécifiquement nous intéresser aux pratiques collectives foncières. En effet à Diohine subsiste une pratique singulière : la gestion collective de l'espace à travers une jachère collective. En questionnant les pratiques foncières induites par la jachère nous avons pu interroger \textit{i)} les interactions entre le rôle d'agriculteur et celui d'éleveur, \textit{ii)} les structures et mécanismes de résolution de conflits, \textit{iii)} les mécanismes de gestion collective de l'espace à proprement parlé, et enfin \textit{iv)} les réseaux de solidarité qui en découle.

  Ce travail se poursuit avec le développement d'un outil de modélisation à base d'agents, co-construits, qui permettra d'explorer avec les acteurs les conséquences de changement de pratique foncière.

\end{spacing}
\end{abstract}
